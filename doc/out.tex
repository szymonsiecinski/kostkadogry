\documentclass[11pt,a4paper]{article}
\usepackage{color}
\usepackage{ifthen}
\usepackage{ifpdf}
\usepackage[headings]{fullpage}
\ifpdf \usepackage[pdftex, pdfpagemode={UseOutlines},bookmarks,colorlinks,linkcolor={blue},plainpages=false,pdfpagelabels,citecolor={red},breaklinks=true]{hyperref}
  \usepackage[pdftex]{graphicx}
  \pdfcompresslevel=9
  \DeclareGraphicsRule{*}{mps}{*}{}
\else
  \usepackage[dvips]{graphicx}
\fi

\newcommand{\entityintro}[3]{%
  \hbox to \hsize{%
    \vbox{%
      \hbox to .2in{}%
    }%
    {\bf  #1}%
    \dotfill\pageref{#2}%
  }
  \makebox[\hsize]{%
    \parbox{.4in}{}%
    \parbox[l]{5in}{%
      \vspace{1mm}%
      #3%
      \vspace{1mm}%
    }%
  }%
}
\newcommand{\refdefined}[1]{
\expandafter\ifx\csname r@#1\endcsname\relax
\relax\else
{$($in \ref{#1}, page \pageref{#1}$)$}\fi}
\date{\today}
\title{KostkaDoGry}
\author{Szymon Sieciński}
\chardef\textbackslash=`\\
\begin{document}
\maketitle
\sloppy
\addtocontents{toc}{\protect\markboth{Contents}{Contents}}
\tableofcontents
\section*{Class Hierarchy}{
\thispagestyle{empty}
\markboth{Class Hierarchy}{Class Hierarchy}
\addcontentsline{toc}{section}{Class Hierarchy}
\subsection*{Classes}
{\raggedright
\hspace{0.0cm} $\bullet$ java.lang.Object {\tiny \refdefined{java.lang.Object}} \\
\hspace{1.0cm} $\bullet$ ActionBarActivity {\tiny } \\
\hspace{2.0cm} $\bullet$ com.example.kostkadogry.About {\tiny \refdefined{com.example.kostkadogry.About}} \\
\hspace{2.0cm} $\bullet$ com.example.kostkadogry.Help {\tiny \refdefined{com.example.kostkadogry.Help}} \\
\hspace{1.0cm} $\bullet$ Activity {\tiny } \\
\hspace{2.0cm} $\bullet$ com.example.kostkadogry.MainActivity {\tiny \refdefined{com.example.kostkadogry.MainActivity}} \\
\hspace{2.0cm} $\bullet$ com.example.kostkadogry.Ustawienia {\tiny \refdefined{com.example.kostkadogry.Ustawienia}} \\
\hspace{1.0cm} $\bullet$ Fragment {\tiny } \\
\hspace{2.0cm} $\bullet$ com.example.kostkadogry.About.PlaceholderFragment {\tiny \refdefined{com.example.kostkadogry.About.PlaceholderFragment}} \\
\hspace{2.0cm} $\bullet$ com.example.kostkadogry.Help.PlaceholderFragment {\tiny \refdefined{com.example.kostkadogry.Help.PlaceholderFragment}} \\
\hspace{1.0cm} $\bullet$ com.example.kostkadogry.Kostka {\tiny \refdefined{com.example.kostkadogry.Kostka}} \\
\hspace{2.0cm} $\bullet$ com.example.kostkadogry.KontrolerKostki {\tiny \refdefined{com.example.kostkadogry.KontrolerKostki}} \\
\hspace{1.0cm} $\bullet$ java.lang.Enum {\tiny \refdefined{java.lang.Enum}} \\
\hspace{2.0cm} $\bullet$ com.example.kostkadogry.Kostka.RodzajKostki {\tiny \refdefined{com.example.kostkadogry.Kostka.RodzajKostki}} \\
}
}
\section{Package com.example.kostkadogry}{
\label{com.example.kostkadogry}\hypertarget{com.example.kostkadogry}{}
\hskip -.05in
\hbox to \hsize{\textit{ Package Contents\hfil Page}}
\vskip .13in
\hbox{{\bf  Classes}}
\entityintro{About}{com.example.kostkadogry.About}{Klasa implementuje zachowanie okna informacji o programie.}
\entityintro{About.PlaceholderFragment}{com.example.kostkadogry.About.PlaceholderFragment}{Widok okna pomocy.}
\entityintro{Help}{com.example.kostkadogry.Help}{Klasa implementuje zachowanie aktywności Pomoc - wyświetla skróconą\phantom{ }instrukcję obsługi aplikacji.}
\entityintro{Help.PlaceholderFragment}{com.example.kostkadogry.Help.PlaceholderFragment}{Widok okna pomocy.}
\entityintro{KontrolerKostki}{com.example.kostkadogry.KontrolerKostki}{Zapewnia wizualizację wyników rzutu kostką, której zachowanie zostało zdefiniowane w klasie Kostka.}
\entityintro{Kostka}{com.example.kostkadogry.Kostka}{Klasa implementująca mechanizm kostki do gry}
\entityintro{Kostka.RodzajKostki}{com.example.kostkadogry.Kostka.RodzajKostki}{Typ wyliczeniowy implementujący rodzaj kostki do gry.}
\entityintro{MainActivity}{com.example.kostkadogry.MainActivity}{Klasa implementująca zachowanie głównej aktywności (okna) programu.}
\entityintro{Ustawienia}{com.example.kostkadogry.Ustawienia}{Klasa implementująca widok ustawień.}
\vskip .1in
\vskip .1in
\subsection{\label{com.example.kostkadogry.About}Class About}{
\hypertarget{com.example.kostkadogry.About}{}\vskip .1in 
Klasa implementuje zachowanie okna informacji o programie.\vskip .1in 
\subsubsection{Declaration}{
\small public class About
\\ {\bf  extends} ActionBarActivity
\refdefined{.ActionBarActivity}}
\subsubsection{Constructor summary}{
\begin{verse}
\hyperlink{com.example.kostkadogry.About()}{{\bf About()}} \\
\end{verse}
}
\subsubsection{Method summary}{
\begin{verse}
\hyperlink{com.example.kostkadogry.About.btnLicenseClicked(View)}{{\bf btnLicenseClicked(View)}} Obsługuje zdarzenie wciśnięcia przycisku Licencja - wyświetla licencję programu.\\
\hyperlink{com.example.kostkadogry.About.btnOKClicked(View)}{{\bf btnOKClicked(View)}} Obsługuje zdarzenie wciśnięcia przycisku OK.\\
\hyperlink{com.example.kostkadogry.About.onCreate(Bundle)}{{\bf onCreate(Bundle)}} Funkcja odpowiada za obsługę uruchomienia programu.\\
\hyperlink{com.example.kostkadogry.About.onCreateOptionsMenu(Menu)}{{\bf onCreateOptionsMenu(Menu)}} Dodaje do menu okna opcje menu.\\
\hyperlink{com.example.kostkadogry.About.onOptionsItemSelected(MenuItem)}{{\bf onOptionsItemSelected(MenuItem)}} Funkcja obsługuje zdarzenie wybrania pozycji w menu okna.\\
\hyperlink{com.example.kostkadogry.About.onPrepareOptionsMenu(Menu)}{{\bf onPrepareOptionsMenu(Menu)}} Obsługuje zdarzenie tworzenia menu okna.\\
\end{verse}
}
\subsubsection{Constructors}{
\vskip -2em
\begin{itemize}
\item{ 
\index{About()}
\hypertarget{com.example.kostkadogry.About()}{{\bf  About}\\}
\texttt{public\ {\bf  About}()
\label{com.example.kostkadogry.About()}}%end signature
}%end item
\end{itemize}
}
\subsubsection{Methods}{
\vskip -2em
\begin{itemize}
\item{ 
\index{btnLicenseClicked(View)}
\hypertarget{com.example.kostkadogry.About.btnLicenseClicked(View)}{{\bf  btnLicenseClicked}\\}
\texttt{public void\ {\bf  btnLicenseClicked}(\texttt{View} {\bf  view})
\label{com.example.kostkadogry.About.btnLicenseClicked(View)}}%end signature
\begin{itemize}
\item{
{\bf  Description}

Obsługuje zdarzenie wciśnięcia przycisku Licencja - wyświetla licencję programu.
}
\item{
{\bf  Parameters}
  \begin{itemize}
   \item{
\texttt{view} -- Bieżący widok}
  \end{itemize}
}%end item
\end{itemize}
}%end item
\item{ 
\index{btnOKClicked(View)}
\hypertarget{com.example.kostkadogry.About.btnOKClicked(View)}{{\bf  btnOKClicked}\\}
\texttt{public void\ {\bf  btnOKClicked}(\texttt{View} {\bf  view})
\label{com.example.kostkadogry.About.btnOKClicked(View)}}%end signature
\begin{itemize}
\item{
{\bf  Description}

Obsługuje zdarzenie wciśnięcia przycisku OK. Obsługa tego zdarzenia skutkuje zamknięciem widoku (okna).
}
\item{
{\bf  Parameters}
  \begin{itemize}
   \item{
\texttt{view} -- Bieżący widok}
  \end{itemize}
}%end item
\end{itemize}
}%end item
\item{ 
\index{onCreate(Bundle)}
\hypertarget{com.example.kostkadogry.About.onCreate(Bundle)}{{\bf  onCreate}\\}
\texttt{protected void\ {\bf  onCreate}(\texttt{Bundle} {\bf  savedInstanceState})
\label{com.example.kostkadogry.About.onCreate(Bundle)}}%end signature
\begin{itemize}
\item{
{\bf  Description}

Funkcja odpowiada za obsługę uruchomienia programu.
}
\item{
{\bf  Parameters}
  \begin{itemize}
   \item{
\texttt{Stan} -- instancji klasy (programu)}
  \end{itemize}
}%end item
\end{itemize}
}%end item
\item{ 
\index{onCreateOptionsMenu(Menu)}
\hypertarget{com.example.kostkadogry.About.onCreateOptionsMenu(Menu)}{{\bf  onCreateOptionsMenu}\\}
\texttt{public boolean\ {\bf  onCreateOptionsMenu}(\texttt{Menu} {\bf  menu})
\label{com.example.kostkadogry.About.onCreateOptionsMenu(Menu)}}%end signature
\begin{itemize}
\item{
{\bf  Description}

Dodaje do menu okna opcje menu.
}
\item{
{\bf  Parameters}
  \begin{itemize}
   \item{
\texttt{menu} -- Tworzone menu programu}
  \end{itemize}
}%end item
\item{{\bf  Returns} -- 
Stan tworzenia opcji menu 
}%end item
\end{itemize}
}%end item
\item{ 
\index{onOptionsItemSelected(MenuItem)}
\hypertarget{com.example.kostkadogry.About.onOptionsItemSelected(MenuItem)}{{\bf  onOptionsItemSelected}\\}
\texttt{public boolean\ {\bf  onOptionsItemSelected}(\texttt{MenuItem} {\bf  item})
\label{com.example.kostkadogry.About.onOptionsItemSelected(MenuItem)}}%end signature
\begin{itemize}
\item{
{\bf  Description}

Funkcja obsługuje zdarzenie wybrania pozycji w menu okna. Wybranie pozycji w menu skutkuje uruchomieniem wskazanego okna (widoku).
}
\item{
{\bf  Parameters}
  \begin{itemize}
   \item{
\texttt{item} -- Pozycja w menu}
  \end{itemize}
}%end item
\item{{\bf  Returns} -- 
Stwierdzenie wyboru opcji 
}%end item
\end{itemize}
}%end item
\item{ 
\index{onPrepareOptionsMenu(Menu)}
\hypertarget{com.example.kostkadogry.About.onPrepareOptionsMenu(Menu)}{{\bf  onPrepareOptionsMenu}\\}
\texttt{public boolean\ {\bf  onPrepareOptionsMenu}(\texttt{Menu} {\bf  menu})
\label{com.example.kostkadogry.About.onPrepareOptionsMenu(Menu)}}%end signature
\begin{itemize}
\item{
{\bf  Description}

Obsługuje zdarzenie tworzenia menu okna. W tym przypadku jest wyłączone - ustawione na \texttt{\small false}.
}
\item{
{\bf  Parameters}
  \begin{itemize}
   \item{
\texttt{menu} -- Tworzone menu programu}
  \end{itemize}
}%end item
\item{{\bf  Returns} -- 
Potwierdzenie tworzenia menu 
}%end item
\end{itemize}
}%end item
\end{itemize}
}
}
\subsection{\label{com.example.kostkadogry.About.PlaceholderFragment}Class About.PlaceholderFragment}{
\hypertarget{com.example.kostkadogry.About.PlaceholderFragment}{}\vskip .1in 
Widok okna pomocy.\vskip .1in 
\subsubsection{Declaration}{
\small public static class About.PlaceholderFragment
\\ {\bf  extends} Fragment
\refdefined{.Fragment}}
\subsubsection{Constructor summary}{
\begin{verse}
\hyperlink{com.example.kostkadogry.About.PlaceholderFragment()}{{\bf About.PlaceholderFragment()}} \\
\end{verse}
}
\subsubsection{Method summary}{
\begin{verse}
\hyperlink{com.example.kostkadogry.About.PlaceholderFragment.onCreateView(LayoutInflater, ViewGroup, Bundle)}{{\bf onCreateView(LayoutInflater, ViewGroup, Bundle)}} \\
\end{verse}
}
\subsubsection{Constructors}{
\vskip -2em
\begin{itemize}
\item{ 
\index{About.PlaceholderFragment()}
\hypertarget{com.example.kostkadogry.About.PlaceholderFragment()}{{\bf  About.PlaceholderFragment}\\}
\texttt{public\ {\bf  About.PlaceholderFragment}()
\label{com.example.kostkadogry.About.PlaceholderFragment()}}%end signature
}%end item
\end{itemize}
}
\subsubsection{Methods}{
\vskip -2em
\begin{itemize}
\item{ 
\index{onCreateView(LayoutInflater, ViewGroup, Bundle)}
\hypertarget{com.example.kostkadogry.About.PlaceholderFragment.onCreateView(LayoutInflater, ViewGroup, Bundle)}{{\bf  onCreateView}\\}
\texttt{public View\ {\bf  onCreateView}(\texttt{LayoutInflater} {\bf  inflater},
\texttt{ViewGroup} {\bf  container},
\texttt{Bundle} {\bf  savedInstanceState})
\label{com.example.kostkadogry.About.PlaceholderFragment.onCreateView(LayoutInflater, ViewGroup, Bundle)}}%end signature
}%end item
\end{itemize}
}
}
\subsection{\label{com.example.kostkadogry.Help}Class Help}{
\hypertarget{com.example.kostkadogry.Help}{}\vskip .1in 
Klasa implementuje zachowanie aktywności Pomoc - wyświetla skróconą\phantom{ }instrukcję obsługi aplikacji.\vskip .1in 
\subsubsection{Declaration}{
\small public class Help
\\ {\bf  extends} ActionBarActivity
\refdefined{.ActionBarActivity}}
\subsubsection{Constructor summary}{
\begin{verse}
\hyperlink{com.example.kostkadogry.Help()}{{\bf Help()}} \\
\end{verse}
}
\subsubsection{Method summary}{
\begin{verse}
\hyperlink{com.example.kostkadogry.Help.btnOKClicked(View)}{{\bf btnOKClicked(View)}} Obsługuje zdarzenie wciśnięcia przycisku OK.\\
\hyperlink{com.example.kostkadogry.Help.onCreate(Bundle)}{{\bf onCreate(Bundle)}} Funkcja odpowiada za obsługę uruchomienia programu.\\
\hyperlink{com.example.kostkadogry.Help.onCreateOptionsMenu(Menu)}{{\bf onCreateOptionsMenu(Menu)}} Dodaje do menu okna opcje menu.\\
\hyperlink{com.example.kostkadogry.Help.onOptionsItemSelected(MenuItem)}{{\bf onOptionsItemSelected(MenuItem)}} Funkcja obsługuje zdarzenie wybrania pozycji w menu okna.\\
\end{verse}
}
\subsubsection{Constructors}{
\vskip -2em
\begin{itemize}
\item{ 
\index{Help()}
\hypertarget{com.example.kostkadogry.Help()}{{\bf  Help}\\}
\texttt{public\ {\bf  Help}()
\label{com.example.kostkadogry.Help()}}%end signature
}%end item
\end{itemize}
}
\subsubsection{Methods}{
\vskip -2em
\begin{itemize}
\item{ 
\index{btnOKClicked(View)}
\hypertarget{com.example.kostkadogry.Help.btnOKClicked(View)}{{\bf  btnOKClicked}\\}
\texttt{public void\ {\bf  btnOKClicked}(\texttt{View} {\bf  view})
\label{com.example.kostkadogry.Help.btnOKClicked(View)}}%end signature
\begin{itemize}
\item{
{\bf  Description}

Obsługuje zdarzenie wciśnięcia przycisku OK. Obsługa tego zdarzenia skutkuje zamknięciem widoku (okna).
}
\item{
{\bf  Parameters}
  \begin{itemize}
   \item{
\texttt{view} -- Bieżący widok}
  \end{itemize}
}%end item
\end{itemize}
}%end item
\item{ 
\index{onCreate(Bundle)}
\hypertarget{com.example.kostkadogry.Help.onCreate(Bundle)}{{\bf  onCreate}\\}
\texttt{protected void\ {\bf  onCreate}(\texttt{Bundle} {\bf  savedInstanceState})
\label{com.example.kostkadogry.Help.onCreate(Bundle)}}%end signature
\begin{itemize}
\item{
{\bf  Description}

Funkcja odpowiada za obsługę uruchomienia programu.
}
\item{
{\bf  Parameters}
  \begin{itemize}
   \item{
\texttt{Stan} -- instancji klasy (programu)}
  \end{itemize}
}%end item
\end{itemize}
}%end item
\item{ 
\index{onCreateOptionsMenu(Menu)}
\hypertarget{com.example.kostkadogry.Help.onCreateOptionsMenu(Menu)}{{\bf  onCreateOptionsMenu}\\}
\texttt{public boolean\ {\bf  onCreateOptionsMenu}(\texttt{Menu} {\bf  menu})
\label{com.example.kostkadogry.Help.onCreateOptionsMenu(Menu)}}%end signature
\begin{itemize}
\item{
{\bf  Description}

Dodaje do menu okna opcje menu.
}
\item{
{\bf  Parameters}
  \begin{itemize}
   \item{
\texttt{menu} -- Tworzone menu programu}
  \end{itemize}
}%end item
\item{{\bf  Returns} -- 
Stan tworzenia opcji menu 
}%end item
\end{itemize}
}%end item
\item{ 
\index{onOptionsItemSelected(MenuItem)}
\hypertarget{com.example.kostkadogry.Help.onOptionsItemSelected(MenuItem)}{{\bf  onOptionsItemSelected}\\}
\texttt{public boolean\ {\bf  onOptionsItemSelected}(\texttt{MenuItem} {\bf  item})
\label{com.example.kostkadogry.Help.onOptionsItemSelected(MenuItem)}}%end signature
\begin{itemize}
\item{
{\bf  Description}

Funkcja obsługuje zdarzenie wybrania pozycji w menu okna. Wybranie pozycji w menu skutkuje uruchomieniem wskazanego okna (widoku).
}
\item{
{\bf  Parameters}
  \begin{itemize}
   \item{
\texttt{item} -- Pozycja w menu}
  \end{itemize}
}%end item
\item{{\bf  Returns} -- 
Stwierdzenie wyboru opcji 
}%end item
\end{itemize}
}%end item
\end{itemize}
}
}
\subsection{\label{com.example.kostkadogry.Help.PlaceholderFragment}Class Help.PlaceholderFragment}{
\hypertarget{com.example.kostkadogry.Help.PlaceholderFragment}{}\vskip .1in 
Widok okna pomocy.\vskip .1in 
\subsubsection{Declaration}{
\small public static class Help.PlaceholderFragment
\\ {\bf  extends} Fragment
\refdefined{.Fragment}}
\subsubsection{Constructor summary}{
\begin{verse}
\hyperlink{com.example.kostkadogry.Help.PlaceholderFragment()}{{\bf Help.PlaceholderFragment()}} \\
\end{verse}
}
\subsubsection{Method summary}{
\begin{verse}
\hyperlink{com.example.kostkadogry.Help.PlaceholderFragment.onCreateView(LayoutInflater, ViewGroup, Bundle)}{{\bf onCreateView(LayoutInflater, ViewGroup, Bundle)}} \\
\end{verse}
}
\subsubsection{Constructors}{
\vskip -2em
\begin{itemize}
\item{ 
\index{Help.PlaceholderFragment()}
\hypertarget{com.example.kostkadogry.Help.PlaceholderFragment()}{{\bf  Help.PlaceholderFragment}\\}
\texttt{public\ {\bf  Help.PlaceholderFragment}()
\label{com.example.kostkadogry.Help.PlaceholderFragment()}}%end signature
}%end item
\end{itemize}
}
\subsubsection{Methods}{
\vskip -2em
\begin{itemize}
\item{ 
\index{onCreateView(LayoutInflater, ViewGroup, Bundle)}
\hypertarget{com.example.kostkadogry.Help.PlaceholderFragment.onCreateView(LayoutInflater, ViewGroup, Bundle)}{{\bf  onCreateView}\\}
\texttt{public View\ {\bf  onCreateView}(\texttt{LayoutInflater} {\bf  inflater},
\texttt{ViewGroup} {\bf  container},
\texttt{Bundle} {\bf  savedInstanceState})
\label{com.example.kostkadogry.Help.PlaceholderFragment.onCreateView(LayoutInflater, ViewGroup, Bundle)}}%end signature
}%end item
\end{itemize}
}
}
\subsection{\label{com.example.kostkadogry.KontrolerKostki}Class KontrolerKostki}{
\hypertarget{com.example.kostkadogry.KontrolerKostki}{}\vskip .1in 
Zapewnia wizualizację wyników rzutu kostką, której zachowanie zostało zdefiniowane w klasie Kostka.\vskip .1in 
\subsubsection{See also}{}

  \begin{list}{-- }{\setlength{\itemsep}{0cm}\setlength{\parsep}{0cm}}
\item{ \texttt{\hyperlink{com.example.kostkadogry.Kostka}{Kostka}} {\small 
\refdefined{com.example.kostkadogry.Kostka}}%end
} 
  \end{list}
\subsubsection{Declaration}{
\small public class KontrolerKostki
\\ {\bf  extends} com.example.kostkadogry.Kostka
\refdefined{com.example.kostkadogry.Kostka}}
\subsubsection{Constructor summary}{
\begin{verse}
\hyperlink{com.example.kostkadogry.KontrolerKostki()}{{\bf KontrolerKostki()}} Tworzy obiekt klasy KontrolerKostka z domyślnymi parametrami (kostka 6-ścienna).\\
\hyperlink{com.example.kostkadogry.KontrolerKostki(com.example.kostkadogry.Kostka.RodzajKostki)}{{\bf KontrolerKostki(Kostka.RodzajKostki)}} Tworzy obiekt klasy Kostka o danym rodzaju.\\
\end{verse}
}
\subsubsection{Method summary}{
\begin{verse}
\hyperlink{com.example.kostkadogry.KontrolerKostki.getIdObrazu()}{{\bf getIdObrazu()}} Funkcja zwraca wygenerowany z zasobów aplikacji ID obrazu odpowiadającego wylosowanej wartości.\\
\end{verse}
}
\subsubsection{Constructors}{
\vskip -2em
\begin{itemize}
\item{ 
\index{KontrolerKostki()}
\hypertarget{com.example.kostkadogry.KontrolerKostki()}{{\bf  KontrolerKostki}\\}
\texttt{public\ {\bf  KontrolerKostki}()
\label{com.example.kostkadogry.KontrolerKostki()}}%end signature
\begin{itemize}
\item{
{\bf  Description}

Tworzy obiekt klasy KontrolerKostka z domyślnymi parametrami (kostka 6-ścienna).
}
\item{{\bf  See also}
  \begin{itemize}
\item{ \texttt{\hyperlink{com.example.kostkadogry.KontrolerKostki}{KontrolerKostki}} {\small 
\refdefined{com.example.kostkadogry.KontrolerKostki}}%end
}
  \end{itemize}
}%end item
\end{itemize}
}%end item
\item{ 
\index{KontrolerKostki(Kostka.RodzajKostki)}
\hypertarget{com.example.kostkadogry.KontrolerKostki(com.example.kostkadogry.Kostka.RodzajKostki)}{{\bf  KontrolerKostki}\\}
\texttt{public\ {\bf  KontrolerKostki}(\texttt{Kostka.RodzajKostki} {\bf  rodzaj})
\label{com.example.kostkadogry.KontrolerKostki(com.example.kostkadogry.Kostka.RodzajKostki)}}%end signature
\begin{itemize}
\item{
{\bf  Description}

Tworzy obiekt klasy Kostka o danym rodzaju.
}
\item{
{\bf  Parameters}
  \begin{itemize}
   \item{
\texttt{rodzaj} -- Rodzaj kostki (liczba ścian kostki)}
  \end{itemize}
}%end item
\item{{\bf  See also}
  \begin{itemize}
\item{ \texttt{\hyperlink{com.example.kostkadogry.KontrolerKostki}{KontrolerKostki}} {\small 
\refdefined{com.example.kostkadogry.KontrolerKostki}}%end
}
\item{ RodzajKostki}
  \end{itemize}
}%end item
\end{itemize}
}%end item
\end{itemize}
}
\subsubsection{Methods}{
\vskip -2em
\begin{itemize}
\item{ 
\index{getIdObrazu()}
\hypertarget{com.example.kostkadogry.KontrolerKostki.getIdObrazu()}{{\bf  getIdObrazu}\\}
\texttt{public int\ {\bf  getIdObrazu}()
\label{com.example.kostkadogry.KontrolerKostki.getIdObrazu()}}%end signature
\begin{itemize}
\item{
{\bf  Description}

Funkcja zwraca wygenerowany z zasobów aplikacji ID obrazu odpowiadającego wylosowanej wartości.
}
\item{{\bf  Returns} -- 
ID obrazu 
}%end item
\end{itemize}
}%end item
\end{itemize}
}
\subsubsection{Members inherited from class Kostka }{
\texttt{com.example.kostkadogry.Kostka} {\small 
\refdefined{com.example.kostkadogry.Kostka}}
{\small 

\vskip -2em
\begin{itemize}
\item{\vskip -1.5ex 
\texttt{public int {\bf  getLiczba}()
}%end signature
}%end item
\item{\vskip -1.5ex 
\texttt{public Kostka.RodzajKostki {\bf  getRodzajKostki}()
}%end signature
}%end item
\item{\vskip -1.5ex 
\texttt{public int {\bf  getZakres}()
}%end signature
}%end item
\item{\vskip -1.5ex 
\texttt{public void {\bf  losujLiczbe}()
}%end signature
}%end item
\item{\vskip -1.5ex 
\texttt{public void {\bf  setRodzajKostki}(\texttt{Kostka.RodzajKostki} {\bf  k})
}%end signature
}%end item
\end{itemize}
}
}
\subsection{\label{com.example.kostkadogry.Kostka}Class Kostka}{
\hypertarget{com.example.kostkadogry.Kostka}{}\vskip .1in 
Klasa implementująca mechanizm kostki do gry\vskip .1in 
\subsubsection{Declaration}{
\small public class Kostka
\\ {\bf  extends} java.lang.Object
\refdefined{java.lang.Object}}
\subsubsection{All known subclasses}{KontrolerKostki\small{\refdefined{com.example.kostkadogry.KontrolerKostki}}}
\subsubsection{Constructor summary}{
\begin{verse}
\hyperlink{com.example.kostkadogry.Kostka()}{{\bf Kostka()}} Tworzy obiekt klasy Kostka o domyślnych parametrach (6-ścienna kostka do gry) i losowej liczbie z rozkładu jednostajnego.\\
\hyperlink{com.example.kostkadogry.Kostka(com.example.kostkadogry.Kostka.RodzajKostki)}{{\bf Kostka(Kostka.RodzajKostki)}} Tworzy obiekt klasy Kostka o danym rodzaju\\
\end{verse}
}
\subsubsection{Method summary}{
\begin{verse}
\hyperlink{com.example.kostkadogry.Kostka.getLiczba()}{{\bf getLiczba()}} Funkcja zwraca wylosowaną liczbę.\\
\hyperlink{com.example.kostkadogry.Kostka.getRodzajKostki()}{{\bf getRodzajKostki()}} Funkcja zwraca aktualny rodzaj kostki\\
\hyperlink{com.example.kostkadogry.Kostka.getZakres()}{{\bf getZakres()}} Funkcja zwraca zakres liczb, które mogą zostać wylosowane.\\
\hyperlink{com.example.kostkadogry.Kostka.losujLiczbe()}{{\bf losujLiczbe()}} Losuje liczbę na kostce do gry z zakresu 1 ÷ N, gdzie N to liczba ścian.\\
\hyperlink{com.example.kostkadogry.Kostka.setRodzajKostki(com.example.kostkadogry.Kostka.RodzajKostki)}{{\bf setRodzajKostki(Kostka.RodzajKostki)}} Funkcja służy do zmiany aktualnego rodzaju kostki.\\
\end{verse}
}
\subsubsection{Constructors}{
\vskip -2em
\begin{itemize}
\item{ 
\index{Kostka()}
\hypertarget{com.example.kostkadogry.Kostka()}{{\bf  Kostka}\\}
\texttt{public\ {\bf  Kostka}()
\label{com.example.kostkadogry.Kostka()}}%end signature
\begin{itemize}
\item{
{\bf  Description}

Tworzy obiekt klasy Kostka o domyślnych parametrach (6-ścienna kostka do gry) i losowej liczbie z rozkładu jednostajnego.
}
\end{itemize}
}%end item
\item{ 
\index{Kostka(Kostka.RodzajKostki)}
\hypertarget{com.example.kostkadogry.Kostka(com.example.kostkadogry.Kostka.RodzajKostki)}{{\bf  Kostka}\\}
\texttt{public\ {\bf  Kostka}(\texttt{Kostka.RodzajKostki} {\bf  rodzaj})
\label{com.example.kostkadogry.Kostka(com.example.kostkadogry.Kostka.RodzajKostki)}}%end signature
\begin{itemize}
\item{
{\bf  Description}

Tworzy obiekt klasy Kostka o danym rodzaju
}
\item{
{\bf  Parameters}
  \begin{itemize}
   \item{
\texttt{rodzaj} -- Rodzaj kostki do gry (liczba ścian kostki)}
  \end{itemize}
}%end item
\end{itemize}
}%end item
\end{itemize}
}
\subsubsection{Methods}{
\vskip -2em
\begin{itemize}
\item{ 
\index{getLiczba()}
\hypertarget{com.example.kostkadogry.Kostka.getLiczba()}{{\bf  getLiczba}\\}
\texttt{public int\ {\bf  getLiczba}()
\label{com.example.kostkadogry.Kostka.getLiczba()}}%end signature
\begin{itemize}
\item{
{\bf  Description}

Funkcja zwraca wylosowaną liczbę.
}
\item{{\bf  Returns} -- 
Wylosowana liczba. 
}%end item
\end{itemize}
}%end item
\item{ 
\index{getRodzajKostki()}
\hypertarget{com.example.kostkadogry.Kostka.getRodzajKostki()}{{\bf  getRodzajKostki}\\}
\texttt{public Kostka.RodzajKostki\ {\bf  getRodzajKostki}()
\label{com.example.kostkadogry.Kostka.getRodzajKostki()}}%end signature
\begin{itemize}
\item{
{\bf  Description}

Funkcja zwraca aktualny rodzaj kostki
}
\item{{\bf  Returns} -- 
rodzaj kostki 
}%end item
\end{itemize}
}%end item
\item{ 
\index{getZakres()}
\hypertarget{com.example.kostkadogry.Kostka.getZakres()}{{\bf  getZakres}\\}
\texttt{public int\lbrack \rbrack \ {\bf  getZakres}()
\label{com.example.kostkadogry.Kostka.getZakres()}}%end signature
\begin{itemize}
\item{
{\bf  Description}

Funkcja zwraca zakres liczb, które mogą zostać wylosowane.
}
\item{{\bf  Returns} -- 
Zakres liczb 
}%end item
\end{itemize}
}%end item
\item{ 
\index{losujLiczbe()}
\hypertarget{com.example.kostkadogry.Kostka.losujLiczbe()}{{\bf  losujLiczbe}\\}
\texttt{public void\ {\bf  losujLiczbe}()
\label{com.example.kostkadogry.Kostka.losujLiczbe()}}%end signature
\begin{itemize}
\item{
{\bf  Description}

Losuje liczbę na kostce do gry z zakresu 1 ÷ N, gdzie N to liczba ścian. Wyjątkiem jest kostka K10, dla której wylosowane zostaną liczby z zakresu 0 ÷ N-1
}
\end{itemize}
}%end item
\item{ 
\index{setRodzajKostki(Kostka.RodzajKostki)}
\hypertarget{com.example.kostkadogry.Kostka.setRodzajKostki(com.example.kostkadogry.Kostka.RodzajKostki)}{{\bf  setRodzajKostki}\\}
\texttt{public void\ {\bf  setRodzajKostki}(\texttt{Kostka.RodzajKostki} {\bf  k})
\label{com.example.kostkadogry.Kostka.setRodzajKostki(com.example.kostkadogry.Kostka.RodzajKostki)}}%end signature
\begin{itemize}
\item{
{\bf  Description}

Funkcja służy do zmiany aktualnego rodzaju kostki.
}
\item{
{\bf  Parameters}
  \begin{itemize}
   \item{
\texttt{k} -- Rodzaj kostki}
  \end{itemize}
}%end item
\end{itemize}
}%end item
\end{itemize}
}
}
\subsection{\label{com.example.kostkadogry.Kostka.RodzajKostki}Class Kostka.RodzajKostki}{
\hypertarget{com.example.kostkadogry.Kostka.RodzajKostki}{}\vskip .1in 
Typ wyliczeniowy implementujący rodzaj kostki do gry. Możliwe wartości: K3, K4, K5, K6, K7, K8, K10, K12, K14, K16, K20, K24, K30, K48.\vskip .1in 
\subsubsection{Declaration}{
\small public static final class Kostka.RodzajKostki
\\ {\bf  extends} java.lang.Enum
\refdefined{java.lang.Enum}}
\subsubsection{Field summary}{
\begin{verse}
\hyperlink{com.example.kostkadogry.Kostka.RodzajKostki.K10}{{\bf K10}} Kostka 10-ścienna.\\
\hyperlink{com.example.kostkadogry.Kostka.RodzajKostki.K12}{{\bf K12}} Kostka 12-ścienna.\\
\hyperlink{com.example.kostkadogry.Kostka.RodzajKostki.K14}{{\bf K14}} Kostka 14-ścienna.\\
\hyperlink{com.example.kostkadogry.Kostka.RodzajKostki.K16}{{\bf K16}} Kostka 14-ścienna.\\
\hyperlink{com.example.kostkadogry.Kostka.RodzajKostki.K20}{{\bf K20}} Kostka 16-ścienna.\\
\hyperlink{com.example.kostkadogry.Kostka.RodzajKostki.K24}{{\bf K24}} Kostka 24-ścienna.\\
\hyperlink{com.example.kostkadogry.Kostka.RodzajKostki.K3}{{\bf K3}} Kostka 3-ścienna.\\
\hyperlink{com.example.kostkadogry.Kostka.RodzajKostki.K30}{{\bf K30}} Kostka 30-ścienna.\\
\hyperlink{com.example.kostkadogry.Kostka.RodzajKostki.K4}{{\bf K4}} Kostka 4-ścienna.\\
\hyperlink{com.example.kostkadogry.Kostka.RodzajKostki.K48}{{\bf K48}} Kostka 48-ścienna.\\
\hyperlink{com.example.kostkadogry.Kostka.RodzajKostki.K5}{{\bf K5}} Kostka 5-ścienna.\\
\hyperlink{com.example.kostkadogry.Kostka.RodzajKostki.K6}{{\bf K6}} Kostka sześcienna (domyślna).\\
\hyperlink{com.example.kostkadogry.Kostka.RodzajKostki.K7}{{\bf K7}} Kostka 7-ścienna.\\
\hyperlink{com.example.kostkadogry.Kostka.RodzajKostki.K8}{{\bf K8}} Kostka 8-ścienna.\\
\end{verse}
}
\subsubsection{Method summary}{
\begin{verse}
\hyperlink{com.example.kostkadogry.Kostka.RodzajKostki.getValue()}{{\bf getValue()}} Funkcja zwracająca liczbę ścian kostki do gry danego rodzaju.\\
\hyperlink{com.example.kostkadogry.Kostka.RodzajKostki.valueOf(java.lang.String)}{{\bf valueOf(String)}} \\
\hyperlink{com.example.kostkadogry.Kostka.RodzajKostki.values()}{{\bf values()}} \\
\end{verse}
}
\subsubsection{Fields}{
\begin{itemize}
\item{
\index{K3}
\label{com.example.kostkadogry.Kostka.RodzajKostki.K3}\hypertarget{com.example.kostkadogry.Kostka.RodzajKostki.K3}{public static final Kostka.RodzajKostki {\bf  K3}}
\begin{itemize}
\item{\vskip -.9ex 
Kostka 3-ścienna.}
\end{itemize}
}
\item{
\index{K4}
\label{com.example.kostkadogry.Kostka.RodzajKostki.K4}\hypertarget{com.example.kostkadogry.Kostka.RodzajKostki.K4}{public static final Kostka.RodzajKostki {\bf  K4}}
\begin{itemize}
\item{\vskip -.9ex 
Kostka 4-ścienna. Wartości liczbowe umieszczone w~wierzchołku kości.}
\end{itemize}
}
\item{
\index{K5}
\label{com.example.kostkadogry.Kostka.RodzajKostki.K5}\hypertarget{com.example.kostkadogry.Kostka.RodzajKostki.K5}{public static final Kostka.RodzajKostki {\bf  K5}}
\begin{itemize}
\item{\vskip -.9ex 
Kostka 5-ścienna. Podstawy zawierają wartości 1 i 5.}
\end{itemize}
}
\item{
\index{K6}
\label{com.example.kostkadogry.Kostka.RodzajKostki.K6}\hypertarget{com.example.kostkadogry.Kostka.RodzajKostki.K6}{public static final Kostka.RodzajKostki {\bf  K6}}
\begin{itemize}
\item{\vskip -.9ex 
Kostka sześcienna (domyślna).}
\end{itemize}
}
\item{
\index{K7}
\label{com.example.kostkadogry.Kostka.RodzajKostki.K7}\hypertarget{com.example.kostkadogry.Kostka.RodzajKostki.K7}{public static final Kostka.RodzajKostki {\bf  K7}}
\begin{itemize}
\item{\vskip -.9ex 
Kostka 7-ścienna. Podstawy zawierają wartości 6 i 7.}
\end{itemize}
}
\item{
\index{K8}
\label{com.example.kostkadogry.Kostka.RodzajKostki.K8}\hypertarget{com.example.kostkadogry.Kostka.RodzajKostki.K8}{public static final Kostka.RodzajKostki {\bf  K8}}
\begin{itemize}
\item{\vskip -.9ex 
Kostka 8-ścienna.}
\end{itemize}
}
\item{
\index{K10}
\label{com.example.kostkadogry.Kostka.RodzajKostki.K10}\hypertarget{com.example.kostkadogry.Kostka.RodzajKostki.K10}{public static final Kostka.RodzajKostki {\bf  K10}}
\begin{itemize}
\item{\vskip -.9ex 
Kostka 10-ścienna. Zawiera wartości od 0 do 9.}
\end{itemize}
}
\item{
\index{K12}
\label{com.example.kostkadogry.Kostka.RodzajKostki.K12}\hypertarget{com.example.kostkadogry.Kostka.RodzajKostki.K12}{public static final Kostka.RodzajKostki {\bf  K12}}
\begin{itemize}
\item{\vskip -.9ex 
Kostka 12-ścienna.}
\end{itemize}
}
\item{
\index{K14}
\label{com.example.kostkadogry.Kostka.RodzajKostki.K14}\hypertarget{com.example.kostkadogry.Kostka.RodzajKostki.K14}{public static final Kostka.RodzajKostki {\bf  K14}}
\begin{itemize}
\item{\vskip -.9ex 
Kostka 14-ścienna.}
\end{itemize}
}
\item{
\index{K16}
\label{com.example.kostkadogry.Kostka.RodzajKostki.K16}\hypertarget{com.example.kostkadogry.Kostka.RodzajKostki.K16}{public static final Kostka.RodzajKostki {\bf  K16}}
\begin{itemize}
\item{\vskip -.9ex 
Kostka 14-ścienna.}
\end{itemize}
}
\item{
\index{K20}
\label{com.example.kostkadogry.Kostka.RodzajKostki.K20}\hypertarget{com.example.kostkadogry.Kostka.RodzajKostki.K20}{public static final Kostka.RodzajKostki {\bf  K20}}
\begin{itemize}
\item{\vskip -.9ex 
Kostka 16-ścienna.}
\end{itemize}
}
\item{
\index{K24}
\label{com.example.kostkadogry.Kostka.RodzajKostki.K24}\hypertarget{com.example.kostkadogry.Kostka.RodzajKostki.K24}{public static final Kostka.RodzajKostki {\bf  K24}}
\begin{itemize}
\item{\vskip -.9ex 
Kostka 24-ścienna.}
\end{itemize}
}
\item{
\index{K30}
\label{com.example.kostkadogry.Kostka.RodzajKostki.K30}\hypertarget{com.example.kostkadogry.Kostka.RodzajKostki.K30}{public static final Kostka.RodzajKostki {\bf  K30}}
\begin{itemize}
\item{\vskip -.9ex 
Kostka 30-ścienna.}
\end{itemize}
}
\item{
\index{K48}
\label{com.example.kostkadogry.Kostka.RodzajKostki.K48}\hypertarget{com.example.kostkadogry.Kostka.RodzajKostki.K48}{public static final Kostka.RodzajKostki {\bf  K48}}
\begin{itemize}
\item{\vskip -.9ex 
Kostka 48-ścienna.}
\end{itemize}
}
\end{itemize}
}
\subsubsection{Methods}{
\vskip -2em
\begin{itemize}
\item{ 
\index{getValue()}
\hypertarget{com.example.kostkadogry.Kostka.RodzajKostki.getValue()}{{\bf  getValue}\\}
\texttt{public int\ {\bf  getValue}()
\label{com.example.kostkadogry.Kostka.RodzajKostki.getValue()}}%end signature
\begin{itemize}
\item{
{\bf  Description}

Funkcja zwracająca liczbę ścian kostki do gry danego rodzaju.
}
\item{{\bf  Returns} -- 
Liczba ścian kostki 
}%end item
\end{itemize}
}%end item
\item{ 
\index{valueOf(String)}
\hypertarget{com.example.kostkadogry.Kostka.RodzajKostki.valueOf(java.lang.String)}{{\bf  valueOf}\\}
\texttt{public static Kostka.RodzajKostki\ {\bf  valueOf}(\texttt{java.lang.String} {\bf  name})
\label{com.example.kostkadogry.Kostka.RodzajKostki.valueOf(java.lang.String)}}%end signature
}%end item
\item{ 
\index{values()}
\hypertarget{com.example.kostkadogry.Kostka.RodzajKostki.values()}{{\bf  values}\\}
\texttt{public static Kostka.RodzajKostki\lbrack \rbrack \ {\bf  values}()
\label{com.example.kostkadogry.Kostka.RodzajKostki.values()}}%end signature
}%end item
\end{itemize}
}
\subsubsection{Members inherited from class Enum }{
\texttt{java.lang.Enum} {\small 
\refdefined{java.lang.Enum}}
{\small 

\vskip -2em
\begin{itemize}
\item{\vskip -1.5ex 
\texttt{protected final Object {\bf  clone}() throws CloneNotSupportedException
}%end signature
}%end item
\item{\vskip -1.5ex 
\texttt{public final int {\bf  compareTo}(\texttt{Enum} {\bf  arg0})
}%end signature
}%end item
\item{\vskip -1.5ex 
\texttt{public final boolean {\bf  equals}(\texttt{Object} {\bf  arg0})
}%end signature
}%end item
\item{\vskip -1.5ex 
\texttt{protected final void {\bf  finalize}()
}%end signature
}%end item
\item{\vskip -1.5ex 
\texttt{public final Class {\bf  getDeclaringClass}()
}%end signature
}%end item
\item{\vskip -1.5ex 
\texttt{public final int {\bf  hashCode}()
}%end signature
}%end item
\item{\vskip -1.5ex 
\texttt{public final String {\bf  name}()
}%end signature
}%end item
\item{\vskip -1.5ex 
\texttt{public final int {\bf  ordinal}()
}%end signature
}%end item
\item{\vskip -1.5ex 
\texttt{public String {\bf  toString}()
}%end signature
}%end item
\item{\vskip -1.5ex 
\texttt{public static Enum {\bf  valueOf}(\texttt{Class} {\bf  arg0},
\texttt{String} {\bf  arg1})
}%end signature
}%end item
\end{itemize}
}
}
\subsection{\label{com.example.kostkadogry.MainActivity}Class MainActivity}{
\hypertarget{com.example.kostkadogry.MainActivity}{}\vskip .1in 
Klasa implementująca zachowanie głównej aktywności (okna) programu. Zawiera interfejs do dodawania/usuwania kostek oraz wybrane kostki do gry. Obok kostek wyświetlana jest bieżąca wartość oraz zakres losowanych liczb.\vskip .1in 
\subsubsection{Declaration}{
\small public class MainActivity
\\ {\bf  extends} Activity
\refdefined{.Activity}}
\subsubsection{Constructor summary}{
\begin{verse}
\hyperlink{com.example.kostkadogry.MainActivity()}{{\bf MainActivity()}} \\
\end{verse}
}
\subsubsection{Method summary}{
\begin{verse}
\hyperlink{com.example.kostkadogry.MainActivity.DieChange(View)}{{\bf DieChange(View)}} Obsługuje zdarzenie naciśnięcia na kostkę.\\
\hyperlink{com.example.kostkadogry.MainActivity.losujLiczbe()}{{\bf losujLiczbe()}} Losuje liczby na kostkach i pokazuje wyniki w postaci obrazów ścian z wartościami wylosowanych liczb.\\
\hyperlink{com.example.kostkadogry.MainActivity.onAccuracyChanged(Sensor, int)}{{\bf onAccuracyChanged(Sensor, int)}} Obsługuje zdarzenie zmiany dokładności czujnika.\\
\hyperlink{com.example.kostkadogry.MainActivity.onActivityResult(int, int, Intent)}{{\bf onActivityResult(int, int, Intent)}} Obsługuje zdarzenie zatwierdzenia ustawień w oknie Ustawienia.\\
\hyperlink{com.example.kostkadogry.MainActivity.onAddDie(View)}{{\bf onAddDie(View)}} Obsługuje procedurę dodawania kostek.\\
\hyperlink{com.example.kostkadogry.MainActivity.onButtonClick(View)}{{\bf onButtonClick(View)}} Obsługuje zdarzenie naciśnięcia na przycisk „Losuj”.\\
\hyperlink{com.example.kostkadogry.MainActivity.onCreate(Bundle)}{{\bf onCreate(Bundle)}} Funkcja odpowiada za obsługę uruchomienia programu.\\
\hyperlink{com.example.kostkadogry.MainActivity.onCreateOptionsMenu(Menu)}{{\bf onCreateOptionsMenu(Menu)}} Tworzy menu programu i wyświetla je na żądanie użytkownika.\\
\hyperlink{com.example.kostkadogry.MainActivity.onOptionsItemSelected(MenuItem)}{{\bf onOptionsItemSelected(MenuItem)}} Funkcja obsługuje zdarzenie wybrania pozycji w menu aplikacji.\\
\hyperlink{com.example.kostkadogry.MainActivity.onPause()}{{\bf onPause()}} Obsługuje zdarzenie przejścia programu do działania w tle.\\
\hyperlink{com.example.kostkadogry.MainActivity.onRemoveDie(View)}{{\bf onRemoveDie(View)}} Obsługuje procedurę usuwania kostek.\\
\hyperlink{com.example.kostkadogry.MainActivity.onResume()}{{\bf onResume()}} Obsługuje zdarzenie ponownego uruchomienia programu.\\
\hyperlink{com.example.kostkadogry.MainActivity.onSensorChanged(SensorEvent)}{{\bf onSensorChanged(SensorEvent)}} Obsługuje zdarzenie zmiany stanu akcelerometru.\\
\end{verse}
}
\subsubsection{Constructors}{
\vskip -2em
\begin{itemize}
\item{ 
\index{MainActivity()}
\hypertarget{com.example.kostkadogry.MainActivity()}{{\bf  MainActivity}\\}
\texttt{public\ {\bf  MainActivity}()
\label{com.example.kostkadogry.MainActivity()}}%end signature
}%end item
\end{itemize}
}
\subsubsection{Methods}{
\vskip -2em
\begin{itemize}
\item{ 
\index{DieChange(View)}
\hypertarget{com.example.kostkadogry.MainActivity.DieChange(View)}{{\bf  DieChange}\\}
\texttt{public void\ {\bf  DieChange}(\texttt{View} {\bf  view})
\label{com.example.kostkadogry.MainActivity.DieChange(View)}}%end signature
\begin{itemize}
\item{
{\bf  Description}

Obsługuje zdarzenie naciśnięcia na kostkę. Skutkiem jest wywołanie okna (widoku) Ustawienia, w którym użykownik może zmienić rodzaj kostki.
}
\item{
{\bf  Parameters}
  \begin{itemize}
   \item{
\texttt{view} -- Bieżący widok (okno) programu.}
  \end{itemize}
}%end item
\end{itemize}
}%end item
\item{ 
\index{losujLiczbe()}
\hypertarget{com.example.kostkadogry.MainActivity.losujLiczbe()}{{\bf  losujLiczbe}\\}
\texttt{public void\ {\bf  losujLiczbe}()
\label{com.example.kostkadogry.MainActivity.losujLiczbe()}}%end signature
\begin{itemize}
\item{
{\bf  Description}

Losuje liczby na kostkach i pokazuje wyniki w postaci obrazów ścian z wartościami wylosowanych liczb.
}
\end{itemize}
}%end item
\item{ 
\index{onAccuracyChanged(Sensor, int)}
\hypertarget{com.example.kostkadogry.MainActivity.onAccuracyChanged(Sensor, int)}{{\bf  onAccuracyChanged}\\}
\texttt{public void\ {\bf  onAccuracyChanged}(\texttt{Sensor} {\bf  sensor},
\texttt{int} {\bf  accuracy})
\label{com.example.kostkadogry.MainActivity.onAccuracyChanged(Sensor, int)}}%end signature
\begin{itemize}
\item{
{\bf  Description}

Obsługuje zdarzenie zmiany dokładności czujnika.
}
\item{
{\bf  Parameters}
  \begin{itemize}
   \item{
\texttt{sensor} -- Rodzaj czujnika}
   \item{
\texttt{accuracy} -- Dokładność}
  \end{itemize}
}%end item
\end{itemize}
}%end item
\item{ 
\index{onActivityResult(int, int, Intent)}
\hypertarget{com.example.kostkadogry.MainActivity.onActivityResult(int, int, Intent)}{{\bf  onActivityResult}\\}
\texttt{public void\ {\bf  onActivityResult}(\texttt{int} {\bf  requestCode},
\texttt{int} {\bf  resultCode},
\texttt{Intent} {\bf  data})
\label{com.example.kostkadogry.MainActivity.onActivityResult(int, int, Intent)}}%end signature
\begin{itemize}
\item{
{\bf  Description}

Obsługuje zdarzenie zatwierdzenia ustawień w oknie Ustawienia. Skutkiem jest ustawienie liczby ścian (rodzaju kostki) wybranej przez użytkownika kostki do gry oraz przekazanie informacji o zakresie losowanych liczb.
}
\item{
{\bf  Parameters}
  \begin{itemize}
   \item{
\texttt{requestCode} -- Kod żądania}
   \item{
\texttt{resultCode} -- Kod odpowiedzi}
   \item{
\texttt{data} -- Dane z okna Ustawienia}
  \end{itemize}
}%end item
\end{itemize}
}%end item
\item{ 
\index{onAddDie(View)}
\hypertarget{com.example.kostkadogry.MainActivity.onAddDie(View)}{{\bf  onAddDie}\\}
\texttt{public void\ {\bf  onAddDie}(\texttt{View} {\bf  view})
\label{com.example.kostkadogry.MainActivity.onAddDie(View)}}%end signature
\begin{itemize}
\item{
{\bf  Description}

Obsługuje procedurę dodawania kostek. Dodanie kostki skutkuje odsłonięciem wartości liczbowej, zakresu liczb i ilustracji przedstawiającej wartość na kostce oraz wyświetleniem komunikatu dla użytkownika.
}
\item{
{\bf  Parameters}
  \begin{itemize}
   \item{
\texttt{view} -- Bieżący widok (okno) programu.}
  \end{itemize}
}%end item
\end{itemize}
}%end item
\item{ 
\index{onButtonClick(View)}
\hypertarget{com.example.kostkadogry.MainActivity.onButtonClick(View)}{{\bf  onButtonClick}\\}
\texttt{public void\ {\bf  onButtonClick}(\texttt{View} {\bf  view})
\label{com.example.kostkadogry.MainActivity.onButtonClick(View)}}%end signature
\begin{itemize}
\item{
{\bf  Description}

Obsługuje zdarzenie naciśnięcia na przycisk „Losuj”.
}
\item{
{\bf  Parameters}
  \begin{itemize}
   \item{
\texttt{view} -- Bieżący widok (okno) programu}
  \end{itemize}
}%end item
\end{itemize}
}%end item
\item{ 
\index{onCreate(Bundle)}
\hypertarget{com.example.kostkadogry.MainActivity.onCreate(Bundle)}{{\bf  onCreate}\\}
\texttt{protected void\ {\bf  onCreate}(\texttt{Bundle} {\bf  savedInstanceState})
\label{com.example.kostkadogry.MainActivity.onCreate(Bundle)}}%end signature
\begin{itemize}
\item{
{\bf  Description}

Funkcja odpowiada za obsługę uruchomienia programu.
}
\item{
{\bf  Parameters}
  \begin{itemize}
   \item{
\texttt{Stan} -- instancji klasy (programu)}
  \end{itemize}
}%end item
\end{itemize}
}%end item
\item{ 
\index{onCreateOptionsMenu(Menu)}
\hypertarget{com.example.kostkadogry.MainActivity.onCreateOptionsMenu(Menu)}{{\bf  onCreateOptionsMenu}\\}
\texttt{public boolean\ {\bf  onCreateOptionsMenu}(\texttt{Menu} {\bf  menu})
\label{com.example.kostkadogry.MainActivity.onCreateOptionsMenu(Menu)}}%end signature
\begin{itemize}
\item{
{\bf  Description}

Tworzy menu programu i wyświetla je na żądanie użytkownika.
}
\item{
{\bf  Parameters}
  \begin{itemize}
   \item{
\texttt{menu} -- Menu programu}
  \end{itemize}
}%end item
\item{{\bf  Returns} -- 
Wartość \texttt{\small true} 
}%end item
\end{itemize}
}%end item
\item{ 
\index{onOptionsItemSelected(MenuItem)}
\hypertarget{com.example.kostkadogry.MainActivity.onOptionsItemSelected(MenuItem)}{{\bf  onOptionsItemSelected}\\}
\texttt{public boolean\ {\bf  onOptionsItemSelected}(\texttt{MenuItem} {\bf  item})
\label{com.example.kostkadogry.MainActivity.onOptionsItemSelected(MenuItem)}}%end signature
\begin{itemize}
\item{
{\bf  Description}

Funkcja obsługuje zdarzenie wybrania pozycji w menu aplikacji. Wybranie pozycji w menu skutkuje uruchomieniem wskazanego okna (widoku).
}
\item{
{\bf  Parameters}
  \begin{itemize}
   \item{
\texttt{item} -- Pozycja w menu}
  \end{itemize}
}%end item
\item{{\bf  Returns} -- 
Stwierdzenie wyboru opcji 
}%end item
\end{itemize}
}%end item
\item{ 
\index{onPause()}
\hypertarget{com.example.kostkadogry.MainActivity.onPause()}{{\bf  onPause}\\}
\texttt{protected void\ {\bf  onPause}()
\label{com.example.kostkadogry.MainActivity.onPause()}}%end signature
\begin{itemize}
\item{
{\bf  Description}

Obsługuje zdarzenie przejścia programu do działania w tle.
}
\end{itemize}
}%end item
\item{ 
\index{onRemoveDie(View)}
\hypertarget{com.example.kostkadogry.MainActivity.onRemoveDie(View)}{{\bf  onRemoveDie}\\}
\texttt{public void\ {\bf  onRemoveDie}(\texttt{View} {\bf  view})
\label{com.example.kostkadogry.MainActivity.onRemoveDie(View)}}%end signature
\begin{itemize}
\item{
{\bf  Description}

Obsługuje procedurę usuwania kostek. Dodanie kostki skutkuje zasłonięciem wartości liczbowej, zakresu liczb i ilustracji przedstawiającej wartość na kostce oraz wyświetleniem komunikatu dla użytkownika.
}
\item{
{\bf  Parameters}
  \begin{itemize}
   \item{
\texttt{view} -- Bieżący widok (okno) programu.}
  \end{itemize}
}%end item
\end{itemize}
}%end item
\item{ 
\index{onResume()}
\hypertarget{com.example.kostkadogry.MainActivity.onResume()}{{\bf  onResume}\\}
\texttt{protected void\ {\bf  onResume}()
\label{com.example.kostkadogry.MainActivity.onResume()}}%end signature
\begin{itemize}
\item{
{\bf  Description}

Obsługuje zdarzenie ponownego uruchomienia programu.
}
\end{itemize}
}%end item
\item{ 
\index{onSensorChanged(SensorEvent)}
\hypertarget{com.example.kostkadogry.MainActivity.onSensorChanged(SensorEvent)}{{\bf  onSensorChanged}\\}
\texttt{public void\ {\bf  onSensorChanged}(\texttt{SensorEvent} {\bf  event})
\label{com.example.kostkadogry.MainActivity.onSensorChanged(SensorEvent)}}%end signature
\begin{itemize}
\item{
{\bf  Description}

Obsługuje zdarzenie zmiany stanu akcelerometru. Losowanie liczby zostanie wywołane, jeżeli bieżące wartości przyspieszeń w osiach X, Y, Z będą większe niż wartość zmiennej NOISE.
}
\item{
{\bf  Parameters}
  \begin{itemize}
   \item{
\texttt{event} -- Zdarzenie związane z czujnikiem}
  \end{itemize}
}%end item
\end{itemize}
}%end item
\end{itemize}
}
}
\subsection{\label{com.example.kostkadogry.Ustawienia}Class Ustawienia}{
\hypertarget{com.example.kostkadogry.Ustawienia}{}\vskip .1in 
Klasa implementująca widok ustawień. Widok wywoływany po naciśnięciu na jedną z widocznych kostek.\vskip .1in 
\subsubsection{Declaration}{
\small public class Ustawienia
\\ {\bf  extends} Activity
\refdefined{.Activity}}
\subsubsection{Constructor summary}{
\begin{verse}
\hyperlink{com.example.kostkadogry.Ustawienia()}{{\bf Ustawienia()}} \\
\end{verse}
}
\subsubsection{Method summary}{
\begin{verse}
\hyperlink{com.example.kostkadogry.Ustawienia.onCreate(Bundle)}{{\bf onCreate(Bundle)}} Metoda obsługuje zdarzenie tworzenia instancji klasy (tworzenia obiektu widoku).\\
\hyperlink{com.example.kostkadogry.Ustawienia.onCreateOptionsMenu(Menu)}{{\bf onCreateOptionsMenu(Menu)}} \\
\hyperlink{com.example.kostkadogry.Ustawienia.onSetClick(View)}{{\bf onSetClick(View)}} Obsługuje zdarzenie naciśnięcia na przycisk Ustaw.\\
\end{verse}
}
\subsubsection{Constructors}{
\vskip -2em
\begin{itemize}
\item{ 
\index{Ustawienia()}
\hypertarget{com.example.kostkadogry.Ustawienia()}{{\bf  Ustawienia}\\}
\texttt{public\ {\bf  Ustawienia}()
\label{com.example.kostkadogry.Ustawienia()}}%end signature
}%end item
\end{itemize}
}
\subsubsection{Methods}{
\vskip -2em
\begin{itemize}
\item{ 
\index{onCreate(Bundle)}
\hypertarget{com.example.kostkadogry.Ustawienia.onCreate(Bundle)}{{\bf  onCreate}\\}
\texttt{protected void\ {\bf  onCreate}(\texttt{Bundle} {\bf  savedInstanceState})
\label{com.example.kostkadogry.Ustawienia.onCreate(Bundle)}}%end signature
\begin{itemize}
\item{
{\bf  Description}

Metoda obsługuje zdarzenie tworzenia instancji klasy (tworzenia obiektu widoku).
}
\item{
{\bf  Parameters}
  \begin{itemize}
   \item{
\texttt{savedInstanceState} -- Stan klasy}
  \end{itemize}
}%end item
\end{itemize}
}%end item
\item{ 
\index{onCreateOptionsMenu(Menu)}
\hypertarget{com.example.kostkadogry.Ustawienia.onCreateOptionsMenu(Menu)}{{\bf  onCreateOptionsMenu}\\}
\texttt{public boolean\ {\bf  onCreateOptionsMenu}(\texttt{Menu} {\bf  menu})
\label{com.example.kostkadogry.Ustawienia.onCreateOptionsMenu(Menu)}}%end signature
\begin{itemize}
\item{
{\bf  Parameters}
  \begin{itemize}
   \item{
\texttt{menu} -- Menu programu}
  \end{itemize}
}%end item
\end{itemize}
}%end item
\item{ 
\index{onSetClick(View)}
\hypertarget{com.example.kostkadogry.Ustawienia.onSetClick(View)}{{\bf  onSetClick}\\}
\texttt{public void\ {\bf  onSetClick}(\texttt{View} {\bf  view})
\label{com.example.kostkadogry.Ustawienia.onSetClick(View)}}%end signature
\begin{itemize}
\item{
{\bf  Description}

Obsługuje zdarzenie naciśnięcia na przycisk Ustaw.
}
\item{
{\bf  Parameters}
  \begin{itemize}
   \item{
\texttt{view} -- Bieżący widok}
  \end{itemize}
}%end item
\end{itemize}
}%end item
\end{itemize}
}
}
}
\end{document}
